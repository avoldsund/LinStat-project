Before we did the experiment we calculated 16 random numbers on a webpage, and we did it in that sequence. This would have worked fine if it wasn't for the lack of uniqueness of the numbers. After having done experiment 3 four times, we sensed something was wrong. Because we were two persons, we could do two experiments at the same time. A little slack was introduced, as we managed to get six experiments in a row using only the small pot.

Not doing the experiments in a specific order is called randomization, i.e. the run order is random. The reason for this is that potential external factors are not confounded with experimental factors.  We also focused on doing genuine run replicates. This means that all external factors should be equal for every experiment, at least the ones we can control. I.e. for every experiment the pot was cooled in cold water, and the lid as well. Afterwards the pots were dried. To get equal temperatures on the cooking plates, the temperature was set to maximum the entire time. In every experiment we used the same amount of water, 2.5 dL.