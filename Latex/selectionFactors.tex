\begin{table}[H] \centering 
  \caption{Factors used in the experiment} 
  \label{factors} 
\begin{tabular}{@{\extracolsep{5pt}} cccccc} 
\\[-1.8ex]\hline 
\hline \\[-1.8ex] 
 & A & B & C & D &\\ 
1 & Lid & Large pot & Soap & Olive oil \\ 
-1 & Without lid & Small pot & No soap & No olive oil\\ 
\hline \\[-1.8ex] 
\end{tabular} 
\end{table} 

We have chosen four factors that we think will be relevant to the problem. Notice that the size of the cooking plate is not the same for the pot sizes. The large pot was set on a large cooking plate, while the small pot was on a small cooking plate.

We had a hope that factor $C$ and $D$ would affect the boiling time individually, but also that the interaction between them would be significant. As olive oil is non polar, it doesn't mix with water. It will, however, mix if the soap is added. We suspected that the olive oil would increase the boiling time because it would affect the surface tension of the water, and maybe make it harder for the water to evaporate, hence delay the boiling state.

The two last factors had a level where low was without anything, while the high level was with two spoons of the substance, about 30 mL.

In this experiment it is easy to control if the factors are at the desired level, because we see if the lid is on, how large the pot is, and if the soap or the olive oil have been added.