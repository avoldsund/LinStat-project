In table \ref{effects1} and table \ref{effects2}, you can see the main effects and the interaction effects. As one can see, the effect of $B$ is a lot larger than the other effects. 
\begin{table}[H] \centering 
  \caption{Effects} 
  \label{effects1} 
\begin{tabular}{@{\extracolsep{5pt}} cccccccc} 
\\[-1.8ex]\hline 
\hline \\[-1.8ex] 
(Intercept) & A1 & B1 & C1 & D1 & A1:B1 & A1:C1 & A1:D1\\ 
$282.125$ & $$-$1.125$ & $$-$99.875$ & $12.875$ & $4.375$ & $$-$3.125$ & $2.625$ & $$-$3.875$\\ 
\hline \\[-1.8ex] 
\end{tabular} 
\end{table}

\begin{table}[H] \centering 
  \caption{Effects} 
  \label{effects2} 
\begin{tabular}{@{\extracolsep{5pt}} cccccccc} 
\\[-1.8ex]\hline 
\hline \\[-1.8ex] 
B1:C1 & B1:D1 & C1:D1 & A1:B1:C1 & A1:B1:D1 & A1:C1:D1 & B1:C1:D1 & A1:B1:C1:D1\\ 
-13.125 & -6.625 & 2.125 & -10.375 & 1.125 & -13.125 & -15.875 & 7.875\\ 
\hline \\[-1.8ex] 
\end{tabular} 
\end{table}

The Anderson-Darling test is a well know test for normality. For our model with third- and fourth-order interactions removed, the test resulted in a p-value of $0.2646$. Since $p > 0.05$, this p-value supports normality of the data.

On the analysis of influence, we used the Cook's Distance as an estimate. In our search for outliners, this test is a good indication of points that merit closer examination in our analysis.

\begin{table}[!htbp] \centering 
  \caption{} 
  \label{} 
\begin{tabular}{@{\extracolsep{5pt}} cccccccc} 
\\[-1.8ex]\hline 
\hline \\[-1.8ex] 
1 & 2 & 3 & 4 & 5 & 6 & 7 & 8 \\ 
0.715 & 0.069 & 0.133 & 0.047 & 0.786 & 0.093 & 0.164 & 0.031\\  \hline
9 & 10 & 11 & 12 & 13 & 14 & 15 & 16 \\
0.216 & 0.014 & 0.0003 & 0.358 & 0.256 & 0.006 & 0.001 & 0.310 \\
\hline \\[-1.8ex] 
\end{tabular} 
\end{table} 