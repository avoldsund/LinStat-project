In table \ref{effects1} and table \ref{effects2}, you can see the main effects and the interaction effects. As one can see, the effect of $B$ is a lot larger than the other effects. This is also shown in the Pareto plot, figure \ref{fig:pareto}. If we should judge from the Pareto plot, $B$ would be the only significant effect.

\begin{table}[H] \centering 
  \caption{Effects} 
  \label{effects1} 
\begin{tabular}{@{\extracolsep{5pt}} cccccccc} 
\\[-1.8ex]\hline 
\hline \\[-1.8ex] 
(Intercept) & A1 & B1 & C1 & D1 & A1:B1 & A1:C1 & A1:D1\\ 
$282.125$ & $$-$1.125$ & $$-$99.875$ & $12.875$ & $4.375$ & $$-$3.125$ & $2.625$ & $$-$3.875$\\ 
\hline \\[-1.8ex] 
\end{tabular} 
\end{table}

\begin{table}[H] \centering 
  \caption{Effects} 
  \label{effects2} 
\begin{tabular}{@{\extracolsep{5pt}} cccccccc} 
\\[-1.8ex]\hline 
B1:C1 & B1:D1 & C1:D1 & A1:B1:C1 & A1:B1:D1 & A1:C1:D1 & B1:C1:D1 & A1:B1:C1:D1\\ 
-13.125 & -6.625 & 2.125 & -10.375 & 1.125 & -13.125 & -15.875 & 7.875\\ 
\hline \\[-1.8ex] 
\end{tabular} 
\end{table}


\begin{figure}[H]
    \centering
    \includegraphics[width=0.6\textwidth]{PDF/paretoPlot.pdf}
    \caption{Pareto plot}
    \label{fig:pareto}
\end{figure}
%
From the main effects plot, figure \ref{fig:mainEff}, it is easy to see that the heating time for the low level is a lot longer than the heating time for the high level. The other three factors show a very small change when going from low to high level.

\begin{figure}[H]
    \centering
    \includegraphics[width=0.6\textwidth]{PDF/mainEffects4factors.pdf}
    \caption{Main effects for y}
    \label{fig:mainEff}
\end{figure}

The interaction effect plot, figure \ref{fig:interaction} does not bring forth any new information. The lines are mostly parallel, except  for the interaction between $B$ and $C$, but it's still only a minor interaction.

\begin{figure}[H]
    \centering
    \includegraphics[width=0.6\textwidth]{PDF/interactionPlot4factors.pdf}
    \caption{Interaction effects}
    \label{fig:interaction}
\end{figure}

On the analysis of influence, we used the Cook's Distance as an estimate. In our search for outliers, this test is a good indication of points that merit closer examination in our analysis. Table \ref{Cook} shows the Cook's Distance for the model where third and fourth-order interaction effects are assumed to be zero. An operational guideline for the D-value is that measurements where $D>1$ should be looked upon more closely. It is seen by \ref{Cook} that this is not the case for our model.

\begin{table}[!htbp] \centering 
  \caption{The Cook's distance}
  \label{Cook} 
\begin{tabular}{@{\extracolsep{5pt}} cccccccc} 
\\[-1.8ex]\hline 
\hline \\[-1.8ex] 
1 & 2 & 3 & 4 & 5 & 6 & 7 & 8 \\ 
0.715 & 0.069 & 0.133 & 0.047 & 0.786 & 0.093 & 0.164 & 0.031\\  \hline
9 & 10 & 11 & 12 & 13 & 14 & 15 & 16 \\
0.216 & 0.014 & 0.0003 & 0.358 & 0.256 & 0.006 & 0.001 & 0.310 \\
\hline \\[-1.8ex] 
\end{tabular} 
\end{table} 

By looking at the residual plot, figure \ref{fig:residual}, a strange pattern emerges. Although it looks like the points are spread out on the vertical axis, they are centered in two areas at the horisontal axis, i.e. they are clustered.

\begin{figure}[H]
    \centering
    \includegraphics[width=0.6\textwidth]{PDF/residualPlot.pdf}
    \caption{Residual plot}
    \label{fig:residual}
\end{figure}

The Anderson-Darling test is a well known test for normality. For our model with third- and fourth-order interactions removed, the test resulted in a p-value of $0.2646$. Since $p > 0.05$, this p-value supports normality of the data.

On the analysis of influence, we used the Cook's Distance as an estimate. In our search for outliers, this test is a good indication of points that merit closer examination in our analysis.

To further investigate the distribution of the errors, a QQ-plot was made, figure \ref{fig:qqPlot}. This shows that it is normally distributed. All the measurements follow the line, except for two points.

\begin{figure}[H]
    \centering
    \includegraphics[width=0.6\textwidth]{PDF/qqPlot.pdf}
    \caption{QQ-plot}
    \label{fig:qqPlot}
\end{figure}

Because the residual plot looked a bit strange, the Box-Cox-test is performed to see if y needs a transformation. But because $\lambda = 1$ is inside the 95 $\%$ confidence interval, no such transformation is required. 

\begin{figure}[H]
    \centering
    \includegraphics[width=0.6\textwidth]{PDF/boxCox.pdf}
    \caption{Box-Cox-plot}
    \label{fig:boxCox}
\end{figure}

\begin{table}[!htbp] \centering 
  \caption{} 
  \label{} 
\begin{tabular}{@{\extracolsep{5pt}} cccccccc} 
\\[-1.8ex]\hline 
\hline \\[-1.8ex] 
1 & 2 & 3 & 4 & 5 & 6 & 7 & 8 \\ 
0.715 & 0.069 & 0.133 & 0.047 & 0.786 & 0.093 & 0.164 & 0.031\\  \hline
9 & 10 & 11 & 12 & 13 & 14 & 15 & 16 \\
0.216 & 0.014 & 0.0003 & 0.358 & 0.256 & 0.006 & 0.001 & 0.310 \\
\hline \\[-1.8ex] 
\end{tabular} 
\end{table}


If the interaction 3-way and 4-way interaction effects are removed, the effects become,
\begin{table}[!htbp] \centering 
  \caption{} 
  \label{} 
\begin{tabular}{@{\extracolsep{5pt}} ccccccccccc} 
\\[-1.8ex]\hline 
\hline \\[-1.8ex] 
(Intercept) & A1 & B1 & C1 & D1 & A1:B1 \\ 
282.125 & -1.125 & -99.875 & 12.875 & 4.375 & -3.125 \\ \hline
A1:C1 & A1:D1 & B1:C1 & B1:D1 & C1:D1 &\\
2.625 & -3.875 & -13.125 & -6.625 & 2.125 \\ 
\hline \\[-1.8ex] 
\end{tabular} 
\end{table}  